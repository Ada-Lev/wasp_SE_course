\documentclass[a4paper,twocolumn]{article}
\usepackage[utf8]{inputenc}
\usepackage{url}
\usepackage{graphicx}

% This assignment requires you to write an essay on Software Engineering for
% AI/AS ("AI Engineering"), how it relates to your research, and how you can
% apply SE principles and tools in your project / on your sub-topic. Please only
% use diagrams/illustrations if necessary, and do not count these towards the
% length of the essay. The essay should be a minimum of 3.5 A4 pages in length,
% and a maximum of 5 A4 pages in length, excluding references. The font should
% be Times New Roman (or equivalent) and 11 point. Be sensible in formatting /
% layout choices.

% The essay should be structured as follows:

% 1. Begin with an introduction/abstract to your research and topic area. This
%    should be a maximum of 400 words, and should provide (just) enough basis so
%    that your answers to the questions below can be understood.

% 2. Select at least 2 principles/ideas/concepts/techniques from Robert's
%    lectures and discuss how they relate to your research and topic area.

% 3. Select at least 2 principles/ideas/concepts/techniques from the guest
%    lectures and discuss how they relate to your research and topic area.



% Your answers to question 4 is the main part of your essay and should be
% approximately 2 A4 pages in length, 1 page per paper.

\title{Behavioral Software Engineering for Code Review Tools\\
\large WASP Software Engineering Assignment} \author{\small Lo Heander,
\texttt{\small lo.heander@cs.lth.se}}
\date{}

\begin{document}

\maketitle

\section*{Introduction}

The DAPPER project's goal is to answer the research question \textbf{``how can
code reviews be made fit for purpose?''}, which is a question that is more and
more relevant today with more and more of software development being done
remotely to some degree. The advent of AI-supported development tools like
Copilot\cite{bird_taking_2023} and ChatGPT\cite{sobania_analysis_2023} also
shifts the focus for developers even more from writing new code to reviewing
code and integrating it into their code bases. 

Code review is a task with a high cognitive
load\cite{pascarella_information_2018}, requiring the reviewer to focus, hold
the code in their memory, look for problems and connections, keep the goal of
the merge request in mind and also take decisions around writing comments and
approving or rejecting the merge request. This limits how long a developer can
keep doing code reviews and how many they can complete before they are fatigued
and the benefits decrease. 

Despite these problems, code review tools have changed surprisingly little since
the first software tool for code view, ICICLE\cite{brothers_icicle_1990}, was
introduced over three decades ago. More modern tools have colored diff-views and
run in the browser, but ICICLE already had many of the core features such as the
central diff view, the ability to add comments, support for annotations by
static analyzers and support for working in a distributed network environment.
This, I think, indicates a superlativist
assumption\cite{green1991comprehensibility} where one tool is assumed to be the
best regardless of context. It favors certain kinds of software development, for
example writing code over UI prototyping and promotes a file-centric view with a
lexical diff over looking at code changes through the lens of exported APIs,
inheritance structures or execution flow. 

To explore these goals and the research question I use an interdisciplinary
approach combining computer science with methods from sociology and psychology
such as ethnography\cite{h_sharp_role_2016}, grounded
theory\cite{adolph_using_2011} and distributed
cognition\cite{hutchins1995cognition}. I believe that to find the answers here,
the practices, tools, interactions and patterns \emph{needs} to be observed from
an interdisciplinary perspective to raise the perspective to a higher level. To
observe the effects and the outcomes and acknowledge that these systems work in
interaction with a team and with their process for version control, continuous
integration, code review, code merging, deployment and maintenance. 

\section*{Software Engineering principles from lectures}

\subsection*{Behavioral Software Engineering}

\subsection*{Quality Assurance in Software Engineering}

\section*{Software Engineering principles from guest lectures}

\subsection*{Levels of assistance and automation}

\section*{Related work}

% 4. Find two full/long papers published in one of the CAIN conferences (3 have
%    been held so far, all linked from
%    https://conf.researchr.org/series/cainLinks to an external site.), download
%    and read them and then write in your assignment, for each paper: a,
%    Describe the core idea(s) of the paper and why it/they are important to the
%    engineering of AI systems b, How the paper relates to your own research c,
%    How your research and its results would fit into a larger AI-intensive
%    software project where one of the core ideas from the paper would benefit
%    the project if applied. Describe both how the paper could help improve the
%    project and how your WASP research would fit into the project. d, Discuss
%    briefly how your research could be potentially adapted/changed to make AI
%    engineering in the project based on the idea of the paper even
%    better/easier.

\subsection*{Design Patterns for AI-based Systems}

``Design Patterns'' is a concept in Software Engineering that describes and
names proven and reusable solutions to frequently occuring
problems~\cite{Gamma2001}. In my first chosen article, Heiland et
al.~\cite{heiland_design_2023} presents an multivocal literature review over both traditional
Software Engineering Design Patterns that are applicable to AI-based system, as
well as new emerging patterns that are unique to, or more relevant to, these
systems. The authors motivation for the study is that since there are
indications that design patterns improves the overall software quality, they are
likely to give similar effects on system quality for AI-based systems as well.
Having a comprehensive overview over possible patterns will help both in future
research of this as well as when implementing and refactoring AI-based systems.

Heiland et al. finds 70 design patterns described both in academic and
practitioner literature. Roughly half of these, 34 patterns, are specific to
AI-based system while the rest are applicable to both AI-based system and
traditional software systems. An interesting finding is that these categories of
patterns address different areas of the system quality. Traditional design
patterns are applied predominately to improve process and implementation aspects
while new AI-system-specific patterns focus on improving security, safety and
deployment aspects.

Techniques and best practices to improve system quality is closely related to my
research field. Because code review in itself is (among other things) a quality
assurance tool, identifying proven design patterns in the code under review
could help the reviewers understand and critique the code. 

Further, my research project also aims to propose improvements in code review
tools that address the shortcomings and frictions found by both other
researchers and in my initial studies. One important avenue to explore here is
how to leverage AI-based system to assist software developers better before,
during and after code reviews. To make these systems perform well, using
established design patterns for both implementation, architecture, security,
etc. would be both useful and important.


\bibliography{references.bib}
\bibliographystyle{ieeetr}

\end{document}
