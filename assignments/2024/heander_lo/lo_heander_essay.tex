\documentclass[a4paper,twocolumn]{article}
\usepackage[utf8]{inputenc}
\usepackage{url}
\usepackage{graphicx}

\title{Behavioral Software Engineering for Code Review Tools\\
\large WASP Software Engineering Assignment}
\author{\small Lo Heander, \texttt{\small lo.heander@cs.lth.se}}
\date{}

\begin{document}

\maketitle

\section*{Introduction}

The DAPPER project's goal is to answer the research question \textbf{``how can code reviews be made
fit for purpose?''}, which is a question that is more and more relevant today with more and more of
software development being done remotely to some degree. The advent of AI-supported development
tools like Copilot\cite{bird_taking_2023} and ChatGPT\cite{sobania_analysis_2023} also shifts the
focus for developers even more from writing new code to reviewing code and integrating it into their
code bases. 

Code review is a task with a high cognitive load\cite{pascarella_information_2018}, requiring the
reviewer to focus, hold the code in their memory, look for problems and connections, keep the goal
of the merge request in mind and also take decisions around writing comments and approving or
rejecting the merge request. This limits how long a developer can keep doing code reviews
and how many they can complete before they are fatigued and the benefits decrease. 

Despite these problems, code review tools have changed surprisingly little since the first software
tool for code view, ICICLE\cite{brothers_icicle_1990}, was introduced over three decades ago. More
modern tools have colored diff-views and run in the browser, but ICICLE already had many of the core
features such as the central diff view, the ability to add comments, support for annotations by
static analyzers and support for working in a distributed network environment. This, I 
think, indicates a superlativist assumption\cite{green1991comprehensibility} where 
one tool is assumed to be the best regardless of context. It favors certain kinds of software
development, for example writing code over UI prototyping and promotes a file-centric
view with a lexical diff over looking at code changes through the lens of exported APIs,
inheritance structures or execution flow. 

To explore these goals and the research question I use an interdisciplinary approach combining
computer science with methods from sociology and psychology such as
ethnography\cite{h_sharp_role_2016}, grounded theory\cite{adolph_using_2011} and distributed
cognition\cite{hutchins1995cognition}. I believe that to find the answers here, the practices,
tools, interactions and patterns \emph{needs} to be observed from an interdisciplinary perspective
to raise the perspective to a higher level. To observe the effects and the outcomes and acknowledge
that these systems work in interaction with a team and with their process for version control,
continuous integration, code review, code merging, deployment and maintenance. 

\section*{Software Engineering principles from lectures}

\subsection*{Behavioral Software Engineering}

\subsection*{Quality Assurance in Software Engineering}

\section*{Software Engineering principles from guest lectures}

\section*{Related work}

\bibliography{references.bib}
\bibliographystyle{ieeetr}

\end{document}
